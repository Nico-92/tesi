\clearpage\null\newpage
\pagebreak
\section{Glossario}
\definizione{Apache}: Vedere Sezione \ref{strumentidilavoro}. \\
\definizione{API}: Application programming interface, sono un insieme di procedure che un sistema software rende accessibile a terzi per interfacciarsi ad esso. \\
\definizione{CakePhp}: framework per la realizzazione di applicazioni web scritto in PHP. È ispirato al concetto di \glossario{MVC}. \\
\definizione{Charta}: Azienda che ha sviluppato il software \tlite. \\
\definizione{Chiave esterna}: Nel contesto dei database relazionali è un vincolo di integrità referenziale tra due o più tabelle. Essa identifica una o più colonne di una tabella (referenziante) che referenzia una o più colonne di un'altra tabella (referenziata). \\
\definizione{Chiave primaria}: Nel modello relazionale della basi di dati è un insieme di attributi che permette di individuare univocamente un record o tupla o ennupla in una tabella o relazione. \\
\definizione{CVS}: Vedere Sezione \ref{strumentidilavoro}. \\
\definizione{Dati personali}: qualunque informazione relativa a persona fisica, persona giuridica, ente od associazione, identificati o identificabili, anche indirettamente, mediante riferimento a qualsiasi altra informazione, ivi compreso un numero di identificazione personale, ad esempio nome, cognome, numeri di telefono, email, indirizzo ed abitazione. \\
\definizione{Dati sensibili}: dati idonei a rivelare l'origine razziale ed etnica, le convinzioni religiose, filosofiche o di altro genere, le opinioni politiche, l'adesione a partiti, sindacati, associazioni od organizzazioni a carattere religioso, filosofico, politico o sindacale, nonché i dati personali idonei a rivelare lo stato di salute e la vita sessuale. \\
\definizione{IAT}: Acronimo di Informazioni assistenza turistica, si tratta degli uffici o sportelli sparsi per la città in cui si possono ottenere informazioni, e acquistare/ricevere la PadovaCard. \\
\definizione{IDE}: Integrated development environment,  è un software che, in periodo di programmazione, aiuta i programmatori nello sviluppo del codice sorgente di un programma. \\
\definizione{IntelliSense}: IntelliSense è un termine generale che rappresenta diverse funzionalità di aiuto
di scrittura del codice: informazioni su classi e metodi, visualizzazione della lista dei metodi disponibili per un tipo, informazioni sui parametri di un metodo e auto-completamento delle parole. \\
\definizione{ISO/IEC 7810:2003}: Standard internazionale stabilito dall'Organizzazione internazionale per la normazione (ISO) che descrive le caratteristiche fisiche delle carte d'identità. Al suo interno sono previsti 4 diversi standard tutti di spessore 0,76 mm, quello scelto per la PadovaCard è ID-1 (85,60 mm x 53,98 mm). \\
\definizione{MD5}: Codifica che prende in input una stringa di lunghezza arbitraria e ne produce in output un'altra a 128 bit. La codifica avviene molto velocemente e l'output restituito è tale per cui è altamente improbabile ottenere con due diverse stringhe in input uno stesso valore hash in output. \\
\definizione{Mockup}: Riproduzione di un oggetto o modello in scala ridotta. In generale viene utilizzato per creare rappresentazioni il cui scopo è dare un’idea visiva, anche molto dettagliata, di come sarà o dovrà essere l’originale. \\
\definizione{MIT}: È una licenza software di tipo permissivo, cioè permette il riutilizzo nel software proprietario sotto la condizione che la licenza sia distribuita con tale software. \\
\definizione{MVC}: Acronimo di Model View Controller, è un Design Pattern architetturale che divide l’applicazione in tre parti interconnesse, separando l’interfaccia utente, la logica e i dati. \\
\definizione{MySQL}: Vedere Sezione \ref{strumentidilavoro}. \\
\definizione{Operatore}: si tratta di operatori degli sportelli \glossario{IAT} o del call center;\\
\definizione{Ore/uomo}: Unità di misura utilizzata per indicare lo sforzo erogato o pianificato per svolgere una attività o un progetto. \\ 
\definizione{OSS}: Software sviluppato da \net per la gestione delle vendite allo \glossario{IAT}. \\
\definizione{Personale}: si tratta del personale delle \glossario{strutture} convenzionate con PadovaCard, si occuperanno di verificarne la validità. \\
\definizione{PHPDocumentor}: Vedere Sezione \ref{strumentidilavoro}. \\
\definizione{Struttura}: si fa riferimento alle strutture convenzionate con PadovaCard, e quindi visitabili dagli \glossario{utenti} che ne sono in possesso \\
\definizione{UML}: Unified Modeling Language, è un linguaggio di modellazione e specifica basato sul paradigma object-oriented. 
È utilizzato per descrivere soluzioni analitiche e progettuali in modo sintetico e comprensibile a un vasto pubblico. Ad oggi si è giunti alla versione 2.0. \\
\definizione{Utente}: si tratta del fruitore della PadovaCard, tipicamente un turista.\\
\definizione{Vivaticket} Piattaforma web basata sul software \tlite a cui gli utenti possono accedere. \\
\definizione{Voucher}: documenti emessi da agenzie di viaggio ai propri clienti, come conferma del diritto a godere, nel loro viaggio, di specifici servizi, in essi indicati e già pagati in precedenza all'agenzia stessa. In questo caso si tratta di un foglio formato A4 stampato dall'hotel che ha il valore di una PadovaCard in quanto su di esso è riportato il codice a barre relativo. \\
