\section{Glossario}
\definizione{CakePhp}: framework per la realizzazione di applicazioni web scritto in PHP. È ispirato al concetto di \glossario{MVC}. \\
\definizione{IAT} Acronimo di Informazioni assistenza turistica, si tratta degli uffici o sportelli sparsi per la città in cui si possono ottenere informazioni, e acquistare/ricevere la PadovaCard. \\
\definizione{MVC}: Acronimo di Model View Controller, è un Design Pattern architetturale che divide l’applicazione in tre parti interconnesse, separando l’interfaccia utente, la logica e i dati. \\
\definizione{Operatore}: si tratta di operatori degli sportelli \glossario{IAT} o del call center;\\
\definizione{OSS}:  \\
\definizione{Personale}: si tratta del personale delle \glossario{strutture} convenzionate con PadovaCard, si occuperanno di verificarne la validità. \\
\definizione{Struttura}: si fa riferimento alle strutture convenzionate con PadovaCard, e quindi visitabili dagli \glossario{utenti} che ne sono in possesso \\
\definizione{Utente}: si tratta del fruitore della PadovaCard, tipicamente un turista.\\
\definizione{Voucher}: documenti emessi da agenzie di viaggio ai propri clienti, come conferma del diritto a godere, nel loro viaggio, di specifici servizi, in essi indicati e già pagati in precedenza all'agenzia stessa. In questo caso si tratta di un foglio formato A4 stampato dall'hotel che ha il valore di una PadovaCard in quanto su di esso è riportato il codice a barre relativo. \\
